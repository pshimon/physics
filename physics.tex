\documentclass[11pt]{report}            
\usepackage[utf8]{inputenc}
\usepackage[english]{babel}
\usepackage[backend=biber,style=numeric,sorting=none]{biblatex}
\usepackage{amsmath}
\usepackage{graphicx}
\usepackage{xcolor}
\usepackage{epigraph}

\title{Thoughts about Physics}
\author{Shimon Panfil, Ph.D.\\
e-mail: shimon.panfil@gmail.com\\
https://pshimon.github.io}

\addbibresource{phys-lib.bib}

\begin{document}                        
\maketitle
\epigraph{Physics is becoming so unbelievably complex that it is taking longer
and longer to train a physicist. It is taking so long, in fact, to train a 
physicist to the place where he understands the nature of physical problems that 
he is already too old to solve them.}{Eugene Paul Wigner}
\tableofcontents                        
\chapter{Mathematical Model }
\label {math-mod}
Physics is about building and investigation of mathematical models  of natural phenomena. 
Proper model should have definite domain where it describes the nature correctly, i.e.
its results do not contradict experimental facts. It should be able to produce answer
to any reasonable question and explain why the question is unreasonable in case it fails.
\chapter{Classical Mechanics}
\label{class-mech}
The best introduction into classical or Newtonian mechanics is written by mathematician V.I. Arnold \cite{arnold-cm}.
All phenomena exist in space (three-dimensional and euclidean) and time (one-dimensional).
There exist coordinate systems (called inertial) possessing the following two properties:
\begin{enumerate}
    \item All the laws of nature at all moments of time are the same in all inertial coordinate systems;
\item  All coordinate systems in uniform rectilinear motion with respect to an inertial one are themselves inertial.
\end{enumerate}
This is called "Galileo's principle of relativity".


The initial state of a mechanical system (the totality of positions and
velocities of its points at some moment of time) uniquely determines all of
its motion. This is called "Newton's principle of determinacy"
\printbibliography[
heading=bibintoc,
title={References}
] 
\clearpage
\end{document}
