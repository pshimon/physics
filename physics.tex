\documentclass[11pt]{book}            
\usepackage[utf8]{inputenc}
\usepackage[russian,english]{babel}
\usepackage{csquotes}
\usepackage[backend=biber,style=numeric,sorting=none]{biblatex}
\usepackage{amsmath}
\usepackage{amsfonts}
\usepackage{graphicx}
\usepackage{xcolor}
\usepackage{epigraph}

\title{Thoughts about Physics}
\author{Shimon Panfil, Ph.D.\\
e-mail: shimon.panfil@gmail.com\\
https://pshimon.github.io}

\addbibresource{phys-lib.bib}

\begin{document}  

\maketitle
\epigraph{Physics is becoming so unbelievably complex that it is taking longer
and longer to train a physicist. It is taking so long, in fact, to train a 
physicist to the place where he understands the nature of physical problems that 
he is already too old to solve them.}{Eugene Paul Wigner}
\tableofcontents                        
\chapter{Metaphysics }
\label {metaphysics}
Physics is about building and investigation of abstract models  of natural phenomena.
My favorite picture consists of "Real World" (Reality, Nature), "Ideal world" (Model space)
and two mapping --- "idealization" which maps "Real World" into "Ideal world" and "realization"
mapping "Ideal world" into "Real World". 

Nature is unique and exists independently of any observer\footnote{"Moon exists even nobody looks at it".
In other case we should investigate a number of "important problems" e.g. what is going
with moon when nobody measures its movement, but some wolf is howling on it.}
. Real word objects behave 
according to Laws of Nature and discovering of these laws is the goal of physics.

Ideal world is populated by models (mainly mathematical). Elements of these models
appear as images of real objects under idealization map. When model is constructed, 
it is investigated and conclusions are mapped back to real world by realization map. 
If we get reasonably accurate description of reality the model is correct and can be foundation 
for other models, if not it is wrong and should be replaced or improved in some way or another.
If the model fails to produce results to be mapped, it simply makes no sense.

Idealization and realization mappings are natural and normally implicit. It is
the point of modelling after all. However some notions like causality, probability, likelihood,
"experimental support", \ldots need special care. Let us take for example  Hempel's raven paradox: "observing
objects that are neither black nor ravens may formally increase the likelihood 
that all ravens are black". 

Indeed suppose that the statement "All ravens are black" is true.
In the form of an 
implication, this can be expressed as: If something is a raven, then it 
is black. Via contraposition, this statement is equivalent to: If something is 
not black, then it is not a raven. So an observation of, say, red cow supports 
the statement that all ravens are black. This looks very strange but actually is quite
straightforward. In formal logic (ideal world) there is now such thing as "experimental support" or
likelihood. Statement may be invalidated by the observation of non black raven, but can not be confirmed in any way. So contraposition means simply that the statement "raven is found which is not black" is equivalent to "something not black is found which is raven". 

On the other hand common sense (real world) does not allow statements about {\em all} ravens. We should say "majority" or "practically all" and  contraposition  disappears.

Below I shall consider topics in classical quantum and statistical mechanics, relativity, and field theory with accent on points which are missed or not clear enough in standard courses.

\chapter{Classical Mechanics}
\label{class-mech}
Classical mechanics tacitly supposes that
\begin{enumerate}
    \item we can measure time and space intervals exactly,
    \item measurements do not disturb the system,
    \item measurements do not take space and time for themselves.
\end{enumerate}

\section{Definitions}
\label{class-mech-def}

Here we follow  \cite{arnold-cm-r}.%\cite{arnold-cm},
%\cite{dubrovin-novikov-fomenko}.
There exist coordinate systems (called inertial) possessing the following two properties:
\begin{enumerate}
    \item All the laws of nature at all moments of time are the same in all inertial coordinate systems;
    \item  All coordinate systems in uniform rectilinear motion with respect to an inertial one are themselves inertial.
\end{enumerate}
This is called "Galileo's principle of relativity".
%\cite{maslov}
We denote the set of all real numbers by $\mathbb{R}$. We denote by $\mathbb{R}^n$ an $n$-dimensional 
real vector space. Affine $n$-dimensional space $\mathbb{A}^n$ is distinguished 
from $\mathbb{R}^n$ in that there is "no fixed origin." Thus the sum of two points
of $\mathbb{A}^n$ is not defined, but their difference is defined and is a vector in $\mathbb{R}^n$.
The group $\mathbb{R}^n$ acts on $\mathbb{A}^n$ as the group of parallel displacements: 
\[a \rightarrow a+b,\quad a,a+b \in \mathbb{A}^n,b\in\mathbb{R}^n\]

A euclidean structure on the vector space $\mathbb{R}^n$ is a positive definite symmetric
bilinear form called a scalar product. The scalar product enables one to
define the distance
\[ \rho(x,y)=||x-y||=\sqrt{(x - y,x- y)} \]
between points of the corresponding affine space $\mathbb{A}^n$. An affine space with this
distance function is called a euclidean space and is denoted by $\mathbb{E}^n$.

The galilean space-time structure consists of the following three elements:
\begin{enumerate}
    \item The universe --- a four-dimensional affine \footnote{Formerly, the universe was provided not with an affine, but with a linear structure (the geocentric system + creation of the universe).} space $\mathbb{A}^4$. 
        The points of $\mathbb{A}^4$ are called world points or events. 
        The parallel displacements of the universe $\mathbb{A}^4$
        constitute a vector space $\mathbb{R}^4$.
    \item Time -—- a linear mapping $t: \mathbb{R}^4 \rightarrow \mathbb{R}$ 
        from the vector space of parallel displacements of the 
        universe to the real "time axis." 
        The time interval from event $a \in \mathbb{A}^4$ 
        to event $b \in \mathbb{A}^4$ is the number $t(b - a)$. 
        If $t(b - a) = 0$, then the events $a$ and $b$ are called simultaneous.
The set of events simultaneous with a given event forms a three-
dimensional affine subspace $\mathbb{A}^3$ in $\mathbb{A}^4$. 
        It is called a space of simultaneous events.
The kernel of the mapping $t$ consists of those parallel displacements of
 $\mathbb{A}^4$ which take some (and therefore every) event into an event simultaneous
with it. This kernel is a three-dimensional linear subspace $\mathbb{R}^3$ of the vector
space $\mathbb{R}^4$.
\item The distance between simultaneous events
    \[\rho(a, b) = ||a - b|| =\sqrt{(a - b,a - b)}\qquad a,b \in \mathbb{A}^3 \]
is given by a scalar product on the space $\mathbb{R}^3$. This distance makes every
space of simultaneous events into a three-dimensional euclidean space $\mathbb{E}^3$.
\end{enumerate}

A space $\mathbb{A}^4$, equipped with a galilean space-time structure, is called a
galilean space. The galilean group is the group of all transformations of a galilean space
which preserve its structure. The elements of this group are called galilean
transformations. Thus, galilean transformations are affine transformations
of $\mathbb{A}^4$ which preserve intervals of time and the distance between simultaneous
events. All galilean spaces are isomorphic to the coordinate space $\mathbb{R} \times \mathbb{R}^3$.

Every galilean transformation of  coordinate space $\mathbb{R} \times \mathbb{R}^3$
can be written in a unique way as the composition of a rotation, a translation,
and a uniform motion ($g = g_1 \circ g_2 \circ g_3$), where
\[g_1(t,\vec x)=(t,\vec x+\vec v t)\qquad \forall t\in \mathbb{R},\vec x \in\mathbb{R}^3\] is uniform motion with velocity $\vec v \in\mathbb{R}^3$,
\[g_2(t,\vec x)=(t+t_0,\vec x+\vec x_0)\qquad \forall t\in \mathbb{R},\vec x \in\mathbb{R}^3\] is translation of the origin, and
\[g_3(t,\vec x)=(t,O\vec x)\qquad \forall t\in \mathbb{R},\vec x \in\mathbb{R}^3\] rotation of the coordinate axes ($O:\mathbb{R}^3 \rightarrow \mathbb{R}^3$ is an orthogonal transformation).

Let $M$ be a set. A one-to-one correspondence 
\[\phi_1: M \rightarrow \mathbb{R} \times \mathbb{R}^3\]
is called a galilean coordinate system on the set $M$. A coordinate system $\phi_2$ moves
uniformly with respect to $\phi_1$  if $\phi_1\circ\phi_2^{-1}$ is a galilean
transformation. The galilean coordinate systems $\phi_2$ and $\phi_1$ give M the same
galilean structure.

A motion in $\mathbb{R}^n$ is a differentiable mapping 
$x: \mathbb{I}\rightarrow \mathbb{R}^n, \mathbb{I} \subset \mathbb{R}$.

The derivative $\dot{x}(t_0)=\frac{dx}{dt}|_{t=t_0}$ is called the velocity vector. 
The second derivative $\ddot{x}(t_0)=\frac{d^2x}{dt^2}|_{t=t_0}$ is called the acceleration vector.
We will assume that the functions we encounter are continuously 
differentiable as many times as necessary.

The image of motion (curve in $\mathbb{R}^n$) called trajectory and  curve in galilean space which appears in some (and therefore every) galilean coordinate system as the graph of a motion, is called a world line.
\section{Newton's equations}
\label{newton-eqs}
Newtonian mechanics studies the motion of a system of point masses (i.e. particles whose
shapes and sizes are inessential, so mass is the only parameter) in three-dimensional euclidean space. 
The initial state of a mechanical system (the totality of positions and
velocities of its points at some moment of time) uniquely determines all of
its motion. In particular, the initial positions and velocities determine the accelerations.
This is called "Newton's principle of determinacy".

Let us consider the mechanical system consisting of $n$ particles 
with masses $m_j,j=1\ldots n$ and word lines $\vec x_j(t)$.
Suppose that $n$ functions  
$F_j:\mathbb{R}^{3 n}\times\mathbb{R}^{3 n}\times\mathbb{R}\rightarrow\mathbb{R}^3,j=1\ldots n$ 
are defined. Then the motion of all particles is determined by equations
\begin{equation}
    m_j \ddot \vec x_j=\vec F_j(\vec x_k,\dot \vec x_k,t) \quad j,k=1\ldots n \label {newton1}
\end{equation}
and initial conditions
\begin{equation}
    \vec x_j(t_0)=\vec y_j\quad\dot \vec x_j(t_0)=\vec v_j  \quad j=1\ldots n \label {newton2}
\end{equation}
Newton's equation \eqref{newton1} is  the basis of mechanics. Functions $\vec F_j$
are called forces they should be determined experimentally for each specific mechanical system.
Galilean invariance is the only restriction put on forces.

Mathematics of  Cauchy problem (\ref{newton1},\ref{newton2}) 
one can find in e.g. \cite{arnold-ode}. Generally the unique solution exists if
all functions are smooth enough.

Note that initial conditions have nothing to do with causality. In many cases we are
interested in solution of equation \eqref{newton1} with final 
conditions (aiming) or in finding the solution which does not 
depend on initial conditions.




\selectlanguage{russian}
\printbibliography[
heading=bibintoc,
title={References}
] 
\clearpage
\end{document}
