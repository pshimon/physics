\documentclass[11pt]{book}            
\usepackage[utf8]{inputenc}
\usepackage[russian,english]{babel}
\usepackage{csquotes}
\usepackage[backend=biber,style=numeric,sorting=none]{biblatex}
\usepackage{amsmath}
\usepackage{amsfonts}
\usepackage{graphicx}
\usepackage{xcolor}
\usepackage{epigraph}

\title{Thoughts about Physics}
\author{Shimon Panfil, Ph.D.\\
e-mail: shimon.panfil@gmail.com\\
https://pshimon.github.io}

\addbibresource{phys-lib.bib}

\begin{document}  

\maketitle
\epigraph{Physics is becoming so unbelievably complex that it is taking longer
and longer to train a physicist. It is taking so long, in fact, to train a 
physicist to the place where he understands the nature of physical problems that 
he is already too old to solve them.}{Eugene Paul Wigner}
\tableofcontents                        
\chapter{Metaphysics }
\label {metaphysics}
Physics is about building and investigation of abstract models  of natural phenomena.
My favorite picture taken from \cite{strugatsky} consists of `Real World' (Reality, Nature), `Ideal world' (Model space)
and two mapping --- `idealization' which maps Real World into Ideal world and `realization'
mapping Ideal world into Real World. 

Nature is unique and exists independently of any observer\footnote{`Moon exists even nobody looks at it' and so does electron.}. Real word objects behave 
according to Laws of Nature and discovering of these laws is the goal of physics.

Ideal world is populated by models (mainly mathematical). Elements of these models
appear as images of real objects under idealization map. When model is constructed, 
it is investigated and conclusions are mapped back to real world by realization map. 
If we get reasonably accurate description of reality the model is correct and can be foundation 
for other models, if not it is wrong and should be replaced or improved in some way or another.
If the model fails to produce results to be mapped, it simply makes no sense.

Idealization and realization mappings are natural and normally implicit. It is
the point of modelling after all. However some notions like causality, probability, likelihood,
`experimental support' need special care. Let us take for example  Hempel's raven paradox: ``observing
objects that are neither black nor ravens may formally increase the likelihood 
that all ravens are black''. 

Indeed suppose that the statement `All ravens are black' is true.
In the form of an 
implication, this can be expressed as: If something is a raven, then it 
is black. Via contraposition, this statement is equivalent to: If something is 
not black, then it is not a raven. So an observation of, say, red cow supports 
the statement that all ravens are black. This looks very strange but actually is quite
straightforward. In formal logic (ideal world) there is now such thing as experimental support or
likelihood. Statement may be invalidated by the observation of non black raven, but can not be confirmed in any way. So contraposition means simply that the statement `raven is found which is not black' is equivalent to `something not black is found which is raven'. 

On the other hand common sense (real world) does not allow statements about {\em all} ravens. We should say majority or practically all and  contraposition  disappears.

Idealization mapping is multi-valued, realization is single-valued (but not one-to-one) mapping \footnote{As V. Korneev (see \cite{strugatsky}) put it: `Transfer of real object to ideal world
is not the same as transfer of ideal object to real world.'}. Physicist often uses more than one model simultaneously and change them on the fly without warning or use some conditions which were not stated explicitly. These tricks irritate mathematicians and confuse engineers. However they are natural parts of physicist's way of thinking.

So what is wrong with standard courses of physics? I see three problems:
\begin{enumerate}
    \item when comparing different models say quantum and classical mechanics the accent is 
        on difference but it should be on similarity and interaction;
    \item adherence to standard (sometimes too restricting) formulation of the problems \footnote{As C. Hunta (see \cite{strugatsky}) said: `I know that solution does not exist, but I want to solve the problem.'};
    \item too much attention to exact solutions of highly oversimplified problems and too little to computational aspects, I see theoretical and computational physics not as different disciplines but as two sides of the same physics.
\end{enumerate}



Below I shall consider in details some topics in classical quantum and statistical mechanics, relativity, and field theory with accent on points mentioned above.

\chapter{Classical Mechanics}
\label{class-mech}

\section{Newton's equations}
\label{newton-eqs}

Classical mechanics considers  space to be three-dimensional  and euclidean $\mathbb{R}^3$ and time 
to be one-dimensional $\mathbb{T}$. Let us call the space $\mathbb{T} \times \mathbb{R}^3$ the (galilean) universe. 
 The  group of galilean transformations acts on universe \cite{arnold-cm}. 
Every galilean transformation can be written in a unique way as the composition of a rotation, a translation,
and a uniform motion ($g = g_1 \circ g_2 \circ g_3$), where
\[g_1(t,\vec x)=(t,\vec x+\vec v t)\qquad \forall t\in \mathbb{R},\vec x \in\mathbb{R}^3\] is uniform motion with velocity $\vec v \in\mathbb{R}^3$,
\[g_2(t,\vec x)=(t+t_0,\vec x+\vec x_0)\qquad \forall t\in \mathbb{R},\vec x \in\mathbb{R}^3\] is translation of the origin, and
\[g_3(t,\vec x)=(t,O\vec x)\qquad \forall t\in \mathbb{R},\vec x \in\mathbb{R}^3\] rotation of the coordinate axes ($O:\mathbb{R}^3 \rightarrow \mathbb{R}^3$ is an orthogonal transformation).

Let $M$ be a set. A one-to-one correspondence 
\[\phi_1: M \rightarrow \mathbb{R} \times \mathbb{R}^3\]
is called a galilean (inertial) coordinate system on the set $M$. A coordinate system $\phi_2$ moves
uniformly with respect to $\phi_1$  if $\phi_1\circ\phi_2^{-1}$ is a galilean
transformation. 

Inertial coordinate systems  possess the following two properties :
\begin{enumerate}
    \item All the laws of nature at all moments of time are the same in all inertial coordinate systems;
    \item  All coordinate systems in uniform rectilinear motion with respect to an inertial one are themselves inertial.
\end{enumerate}
This is called `Galileo's principle of relativity'.

The simplest object of classical mechanics is point mass (particle) i.e. object whose shape, sizes and internal structure are inessential, so mass (positive scalar $m$) is the only parameter. The state of a particle is fully determined by position ($\vec x \in\mathbb{R}^3$) and velocity ($\vec v \in\mathbb{R}^3$). More accurately: A motion in $\mathbb{R}^n$ is a differentiable mapping 
$\vec x: \mathbb{I}\rightarrow \mathbb{R}^3, \mathbb{I} \subset \mathbb{R}$. The derivative $\dot \vec{x}(t_0)=\frac{d\vec x}{dt}|_{t=t_0}$ is called the velocity vector. The derivative $\ddot \vec x(t_0)=\frac{ d^2 \vec x}{dt^2}|_{t=t_0}$ is called the acceleration vector. The image of motion (curve in $\mathbb{R}^3$) called trajectory and  curve in $\mathbb{T} \times \mathbb{R}^3$ which appears  as the graph of a motion, is called a world line.

The initial state of a mechanical system (the totality of positions and
velocities of its points at some moment of time) uniquely determines all of
its motion. In particular, the initial positions and velocities determine the accelerations.
This is called `Newton's principle of determinacy'.

Let us consider the mechanical system consisting of $n$ particles 
with masses $m_j,j=1\ldots n$ and word lines $\vec x_j(t)$.
Suppose that $n$ functions  
$F_j:\mathbb{R}^{3 n}\times\mathbb{R}^{3 n}\times\mathbb{R}\rightarrow\mathbb{R}^3,j=1\ldots n$ 
are defined. Then the motion of all particles is determined by equations
\begin{equation}
    m_j \ddot \vec x_j=\vec F_j(\vec x_k,\dot \vec x_k,t) \quad j,k=1\ldots n \label {newton1}
\end{equation}
and initial conditions
\begin{equation}
    \vec x_j(t_0)=\vec y_j\quad\dot \vec x_j(t_0)=\vec v_j  \quad j=1\ldots n \label {newton2}
\end{equation}
Newton's equation \eqref{newton1} is  the basis of mechanics. Functions $\vec F_j$
are called forces they should be determined  for each specific mechanical system. If our system
is isolated i.e. includes explicitly all interacting particles, then galilean invariance puts some restrictions on forces.

Actually in most cases it is more convenient to write
\eqref{newton1} as a system of first order equations. 
To do so we should express the  velocity vector explicitly.

Mathematics of  Cauchy problem (\ref{newton1},\ref{newton2}) 
one can find in e.g. \cite{arnold-ode}. Generally the unique solution exists if
all functions are smooth enough. However smoothness may be violated in physically
important cases, anyway functions $x_j(t)$ should be continuous, $\vec v_j(t)$ may
have finite jumps as a results of strong force acting for very short time (e.g. 
elastic collision).

Note that initial conditions have nothing to do with causality. In many cases we are
interested in solution of equation \eqref{newton1} with final 
conditions (aiming) or in finding the solution which does not 
depend on initial conditions. Very important variational problem when
conditions are set at two different times may have more than one solutions
or no solutions at all.

\section{Foundations}
\label{cm-foundations}
\subsection{Global Simultaneity}
Time is universal and global in classical mechanics. State of the system is determined by
positions and velocities at fixed time. Every particle has its own coordinates however time
is common. We can always measure time  exactly and instantly without any influence 
on mechanical systems. Time is continuous, discrete time would lead to finite difference
equations instead of differential ones. So dynamics would be undefined, since one always can
add periodical function to solution.

\subsection{Instant Interaction}
All interactions are instant i.e. depend on particle states at one and the same moment of time.
In particular it means that all accelerations and hence evolution is determined by one state (not necessary initial one). It is possible to think that any interaction takes small but finite period of time. This supposition will lead to functional equations instead of differential ones.

\subsection{Exactness}
We can find out and set up exact value of any variable at any time instantly without disturbing
the system in any way. In particular we can set exactly positions and velocities which means 
among other things that space is continuous.







%\selectlanguage{russian}
\printbibliography[
heading=bibintoc,
title={References}
] 
\clearpage
\end{document}
